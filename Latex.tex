\documentclass[10pt]{beamer}
\mode<presentation>
\usepackage{graphicx}
%\input{psfig.tex}
%\input epsf
%\usepackage{graphics, lscape, amsmath, amsbsy, amstext}
\usepackage{latexsym, epsfig, verbatim,  amsmath, amssymb, lscape}
\usetheme{Warsaw} %{Berlin} {Frankfurt}
\usepackage[english]{babel}
\usepackage[latin1]{inputenc}
\usepackage{color}
%\usecolortheme{rose}
%\usepackage{pgfpages}

\setbeamercovered{transparent}
\def\softness{0.4}
\definecolor{softblue}{rgb}{\softness,\softness,1}
%\setbeamerfont{title}{shape=\itshape,family=\rmfamily}
\setbeamercolor{title}{fg=red!80!black,bg=red!20!white}
%\setbeamercolor{headline}{bg=white}
\setbeamercolor{background canvas}{bg=}
%\setbeamertemplate{mini frames}[box]
%{\usebeamercolor{headline}}
%\setbeamertemplate{headline}[left=white,right=blue]
%{%
%\begin{beamercolorbox}{section in head/foot}
%\vskip2pt\insertnavigation{\paperwidth}\vskip2pt
%\end{beamercolorbox}%
%}
%\setbeamertemplate{footline}[page number]
\setbeamertemplate{navigation symbols}{}
\setbeamerfont{smallsize}{size=\small}
\setbeamerfont{tinysize}{size=\tiny}

\beamertemplatefootpagenumber
\def\by{\mbox{\boldmath $y$}}
\def\bv2{\mbox{\boldmath $v^2$}}
\def\bA{\mbox{\boldmath $A$}}
\def\bb{\mbox{\boldmath $b$}}
\def\balpha{\mbox{\boldmath $\alpha$}}
\def\bM{\mbox{\boldmath $M$}}
\def\bsigma2{\mbox{\boldmath $\sigma^2$}}
\def\ba{\mbox{\boldmath $a$}}

\title[Wuhan University 2010]
{Latex}
%\subtitle{Ph.D. Dissertation Proposal}
%\author{\sc LIU Li }

\institute{\sc School of Mathematics and Statistics, \\
\vspace{0.1cm} \sc Wuhan University }

\date
%{\small Feburary, 2012}

%\setbeameroption{show notes}
\begin{document}


\frame[plain]{\titlepage}



\section{Introduction}





\section{Introduction to \LaTeX{}}
\frame{ \frametitle{Why learn \LaTeX{}}
\begin{itemize}

    \item easy to produce professional-looking mathematical formulas

    \item easy to label equations, citations, figures, tables, etc.
    to automate cross-referencing

\item can be used on any type of computer

\item freely available

\item {\bf .tex} files are plain text: can be produced with any text
editor and emailed to co-authors

\item {{\bf .dvi} files produced in \LaTeX{} processing can be viewed
on screen and printed on almost all kinds of printers\\
-- {\bf .dvi} is short for {\it device independent}}

\item particularly useful to academics: many journals now want
electronic submission of manuscripts in \LaTeX{} format
\end{itemize}
}
%----------------------------------------------------------------------------------

\frame{ \frametitle{Where can we find it?}
\begin{itemize}
  \item http://www.ctex.org/LaTeX
  \item editor: Winedit
\end{itemize}

}
%----------------------------------------------------------------------------------

\frame{ \frametitle{Using \LaTeX{} in the laboratories}

1. prepare source file: $<$name$>$.tex in text editor

\begin{itemize}

    \item filename extension must be {\bf .tex}

\end{itemize}

2. spell check source file: {\bf ispell $<$name$>$.tex}

3. process source file: {\bf latex $<$name$>$.tex}

4. check that the following files exist: {\bf $<$name$>$.log,
$<$name$>$.aux, $<$name$>$.dvi}

5. view .dvi file: {\bf ispell xdvi $<$name$>$}

6. convert .dvi file to Postscript (.ps) file

7. view .ps file

8. (optional) convert .ps file to .pdf file

9. (optional) view .pdf file

10. {\bf .dvi} and especially {\bf .ps} and {\bf .pdf} files can be
large,  so smart to delete them when you're done using them

\begin{itemize}

    \item don't delete the {\bf .tex} file!

\end{itemize}
}


\frame{ \frametitle{Basic \LaTeX}
\begin{itemize}

    \item {lines that must appear in {\it every} \LaTeX\ document\\
    {\bf $\backslash$documentclass\{$<$class$>$\}}\\
    {\bf $\backslash$begin\{document\}}\\
    {\bf $\backslash$end\{document\}}
    }

    \item {classes of document producing different default formats\\

    -- article\\

    -- report\\

    -- book\\

    -- slides\\

    -- letter
    }


\end{itemize}
}


\frame[plain]{\frametitle{Sample .tex file}

\% articletemplate.tex

\vskip0.3cm

\noindent $\backslash$documentclass[12pt]\{article\} \% statement
required; 12 pt
 opti

 \vskip0.3cm

\noindent \% preamble\\
 $\backslash$usepackage[dvips]\{graphics\}\\
 $\backslash$usepackage\{amssymb,amsmath\}\\
 $\backslash$makeindex

\vskip0.3cm

\noindent\% start document\\
$\backslash$begin\{document\} \% required

\vskip0.3cm

 \noindent \% article heading

 $\backslash$title\{Example of $\backslash$LaTeX$\backslash$ document\}

 $\backslash$author\{Your Name\}

 $\backslash$date\{$\backslash$today\}

 $\backslash$maketitle

 \vskip0.3cm
 \noindent \% $\backslash$tablecontents

 \vskip0.3cm
 $\backslash$begin\{abstract\}\\
 \hspace{20pt}This article demonstrate usage of basic
 $\backslash$LaTeX$\backslash$ feature

 $\backslash$end\{abstract\}
}


 \frame{\frametitle{Sample .tex file}

\noindent$\backslash$section\{Automatic paragraph formatting\}
$\backslash$label\{autoform\}

\vskip0.3cm

This is paragraph 1.

\vskip0.3cm

To start a new paragraph, simply leave one or more blank lines.
$\backslash$LaTeX$\backslash$ will do the indenting automatically.
$\backslash$LaTeX$\backslash$ automatically indents the first line
in all paragraphs except the first in a section.

\vskip0.3cm

It doesn't matter how many spaces \hspace{20pt} you leave in between
words or where you break lines---\\
$\backslash$LaTeX$\backslash$ considers a carriage return (where you
pressed "Enter") as just another space between words.

}

\frame{\frametitle{Sample .tex file}

\noindent$\backslash$section\{Special characters in
$\backslash$LaTeX\} $\backslash$label\{specchar\}

\vskip0.3cm

The following characters are special codes in $\backslash$LaTeX:
$\backslash$\&, $\backslash$\$, $\backslash$\~~, $\backslash$\_ ,
$\backslash$\{, $\backslash$\}, $\backslash$\#, $\backslash$\^~. To
print one of these characters literally, you must put a backslash
before it. The backslash itself obviously also is a special
character.

\vskip0.3cm


\noindent$\backslash$subsection\{$\backslash$\%
\}$\backslash$label\{pcntsign\}

\vskip0.3cm

 The percent sign is used to insert comments in a \{$\backslash$tt
 .tex\} file. It tells $\backslash$LaTeX$\backslash$ to ignore
 everything that comes after it on the line.

}

\frame{\frametitle{Sample .tex file}

\noindent$\backslash$section\{Mathematical expressions \}
$\backslash$label\{mathexp\}

\vskip0.3cm

Mathematical expressions may be included in the text of a paragraph
by putting a dollar sign at the beginning and the end of each, like
this: \$e=mc\^~2\$. The special backslash character is printed with
\$$\backslash$backslash\$.

 \vskip0.3cm

Alternatively, a mathematical expression may be set off on its own
line like this:

$\backslash$[

\hspace{20pt} e=mc\^~2

$\backslash$]

\vskip0.3cm

Also,  $\backslash$LaTeX$\backslash$ can number equations and keep
 track of the numbering for you, like this:
\vskip0.3cm

$\backslash$begin\{equation\}$\backslash$label\{equa\}

\hspace{20pt} e=mc\^~2

$\backslash$end\{equation\}

}

\frame{\frametitle{Sample .tex file}

\noindent {\bf Remark:}\{Some math in \LaTeX:\}

\begin{itemize}

 \item {Greek letters\\
\hspace{20pt} \$$\backslash$theta\$, \$$\backslash$Theta\$, \$$\backslash$omega\$, and \$$\backslash$Omega\$\\
\hspace{20pt} $\theta, \Theta, \omega$ and $\Omega$\\
\hspace{20pt} \$$\backslash$mbox\{$\backslash$boldmath\$
$\backslash$theta\$, \}\$\\
\hspace{20pt} $\mbox{\boldmath $\theta$}$
 }

 \item{
aligned equations\\
\hspace{20pt}$\backslash$begin\{eqnarray\}\\
\hspace{25pt} \{$\backslash$bf y\} \& $\backslash$sim \& N(X
\$$\backslash$beta\$, \$$\backslash$Sigma\$) $\backslash\backslash$
\\
\hspace{20pt} $\backslash$Sigma \&=\&\\
\hspace{35pt}$\backslash$left[ $\backslash$begin\{array\}\{cc\} \\
\hspace{45pt} $\backslash$sigma\_\{11\} \& $\backslash$sigma\_\{12\}
$\backslash\backslash$ \\
\hspace{45pt} $\backslash$sigma\_\{21\} \&
$\backslash$sigma\_\{22\}\\
\hspace{35pt} $\backslash$end\{array\}
$\backslash$right] \\
\hspace{20pt}$\backslash$end\{eqnarray\}

\begin{eqnarray}
{\bf y}&\sim& N(X\beta, \Sigma)\\
\Sigma &=& \left[
\begin{array}{cc}
\sigma_{11} & \sigma_{12}\\
\sigma_{21} & \sigma_{22}
\end{array}
\right]
\end{eqnarray}
 }

\item special symbols

\end{itemize}
}

\frame{\frametitle{Sample .tex file}


\hspace{20pt}$\backslash$begin\{eqnarray*\}
\% asterisk suppresses numbering\\
\hspace{25pt} y \&=\& $\backslash$sqrt\{ $\backslash$frac\{q\}\{r\}
\} $\backslash\backslash$
\\
\hspace{25pt}i=1, $\backslash$ldots,
n\\
\hspace{20pt}$\backslash$end\{eqnarray*\}

\begin{eqnarray*}
y&=&\sqrt{\frac{q}{r}}\\
i=1,\ldots,n
\end{eqnarray*}
}

\frame{\frametitle{Sample .tex file}


{\bf ex.} Try to print:


1. $\displaystyle \int_a^b f(x)dx$

2. $\displaystyle \sum\limits_{i,j,k=1}^n x_{i_{j_k}}$

3. $\displaystyle \frac{f(x)}{g(x)}$

4. $\displaystyle \lim\limits_{x\to x_0}=A$

5. $\displaystyle \sqrt[3]{x^4-3x+1}$

6. $\displaystyle \iint_\Omega f(x,y)dxdy$

7. $$\left|\begin{array}{ccc} 1&6&9\\
7&90&\sin(x)\\
9&\psi(x)&\log(x) \end{array}\right|$$ }

\frame{\frametitle{Sample .tex file}

\noindent$\backslash$section\{Using labels\}
$\backslash$label\{labels\}

\vskip0.3cm

Because we have used labels on our sections and equation, we can
refer to them without having to remember the numbers ourselves. For
example, equation ($\backslash$ref\{equa\}) appeared in section
$\backslash$ref\{mathexp\}. This capability is particularly handy
when we add sections or equations, or reorganize a document.
}

\frame{\frametitle{Sample .tex file}

\noindent$\backslash$section\{Environments\}
$\backslash$label\{envi\}

\vskip0.3cm

\noindent$\backslash$subsection\{Lists\}

\vskip0.3cm

 $\backslash$LaTeX$\backslash$ has two list environments:

\vskip0.3cm

\hspace{20pt}$\backslash$begin\{itemize\}\\
\hspace{30pt}$\backslash$item bulleted lists\\
\hspace{30pt}$\backslash$item numbered lists\\
\hspace{40pt}$\backslash$begin\{enumerate\}\\
\hspace{50pt}$\backslash$item differ from bulleted lists in the environment name\\
\hspace{50pt}$\backslash$item lists can be nested within lists\\
\hspace{40pt}$\backslash$end\{enumerate\}\\
\hspace{20pt}$\backslash$end\{itemize\}\\
}

\frame{\frametitle{Including graphics files in a \LaTeX\ file}

\begin{itemize}

\item{
include in the preamble\\
\hspace{30pt} $\backslash$usepackage[dvips]\{graphics\} }

\item include in the body of the document
\end{itemize}

\vskip0.3cm


\hspace{20pt}$\backslash$begin\{figure\}[$<$h,t,or b$>$]\\
\hspace{25pt}$\backslash$begin\{center\}\\
\hspace{30pt}$\backslash$scalebox\{$<$size$>$\}\{$\backslash$includegraphics\{$<$filename.ps or filename.eps$>$\}\}\\
\hspace{25pt}$\backslash$end\{center\}\\
\hspace{25pt}$\backslash$caption\{$<$caption$>$\}\\
\hspace{20pt}$\backslash$end\{figure\}

-- letter h,t and b mean the same as in table

-- $<$size$>$ in {\tt scalebox} command means what multiple of size
of original figure to use (e.g. 0.5 for half)

-- graphics do not have to be put in {it figure} environment

-- {it figure} environment makes graph "floating" and enables adding
caption
}

\frame{\frametitle{Adding a bibliography}

\begin{itemize}

\item built-in bibliographic capabilities in \LaTeX\ enable matching
references in the body of the text to entries in the bibliography

\item creating the bibliography at the end of the article

\vskip0.3cm


\hspace{20pt}$\backslash$begin\{thebibliography\}\{9\}  \% 9 if $<10$ items in biblio\\
\hspace{30pt}$\backslash$bibitem\{Cow96\}\\
\hspace{40pt}Cowles,M.K. (1996)\\
\hspace{40pt}Accelerating Markov chain Monte Carlo convergence\\
\hspace{40pt}for cumulative-link generalized linear models.\\
\hspace{40pt}\{$\backslash$em Statistics and Computing\},
\{$\backslash$bf 6\}, 101--111.\\
\hspace{20pt}$\backslash$end\{thebibliography\}

\item citing references in the body of the text

\vskip0.3cm

{\it Blocking may solve the problem of slow convergence in a Gibbs
sampler for a cumulative link GLM as shown in
$\backslash$cite\{Cow96\}}.

\end{itemize}
 }

 \frame{\frametitle{creating slide}

Special document class for creating slide presentations with
Powerpoint-like feature: beamer

\vskip0.3cm

$\backslash$documentclass[cjk]\{beamer\}

$\backslash$usepackage\{CJK\}

$\backslash$usetheme\{Warsaw\}

\vskip0.3cm

$\backslash$begin\{document\}\\
\hspace{20pt}   $\backslash$begin\{CJK*\}\{GBK\}\{kai\}\\
\hspace{30pt}       $\backslash$title\{Example Presentation Created
$\backslash\backslash$ \\
\hspace{35pt}              with the Beamer
Package$\backslash\backslash$\}\\
\hspace{30pt}       $\backslash$author\{Liu Li\}\\
\hspace{30pt}       $\backslash$date\{$\backslash$today\}

\vskip0.3cm

\hspace{30pt}       $\backslash$frame\{$\backslash$titlepage\}\\
\hspace{30pt}       $\backslash$section*\{outline\}\\
\hspace{30pt}       $\backslash$frame\{$\backslash$tableofcontents\}

\vskip0.3cm

\hspace{30pt}       $\backslash$section\{Introduction\}\\
\hspace{30pt}       $\backslash$subsection\{Overview of the Beamer
Class\}\\
\hspace{30pt}       $\backslash$frame\{\\
\hspace{35pt}          $\backslash$frametitle\{Features of the
Beamer Class\}

\vskip0.3cm

\hspace{35pt}          $\backslash$begin\{itemize\}\\
\hspace{35pt}          $\backslash$item$<1->$ Normal
$\backslash$LaTeX$\backslash$ class\\
\hspace{35pt}          $\backslash$item$<2->$ Easy overlays\\
\hspace{35pt}          $\backslash$item$<3->$ No external programs
needed\\
\hspace{35pt}          $\backslash$end\{itemize\}\\
\hspace{30pt}       \}

\vskip0.3cm

\hspace{20pt}    $\backslash$end\{CJK*\}

$\backslash$end\{document\}
 }

 \frame{\frametitle{creating slide}

\begin{itemize}

\item documentclass must be {\it beamer}, {\it cjk} in the square
bracket allows Chinese to work

\item contents of slides should be put in $\backslash$frame\{\}, and
remember: {\tt a frame is a slide, and DON'T put too many contents
in a slide}

\item $<-1>$ behind item make the slide pause. We can also use
$\backslash$pause to reach the similar effect.

\item a relative web site: {\it http:// latex-beamer.sourceforge.net}

\end{itemize}
 }


\end{document}
